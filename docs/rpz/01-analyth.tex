\chapter{Аналитический раздел}

В этом разделе будут представлены описание объектов, а также обоснован выбор алгоритмов, которые будут использован для ее визуализации.

\section{Описание объектов сцены}

Сцена состоит из источника света, цилиндра, жидкости, стержня и плоскости.

Источник света представляет собой материальную точку, пускающую лучи света во все стороны (если источник расположен в бесконечности, то лучи идут параллельно). Источником света в программе является вектор.
 
Цилиндр --- это тонкостенный прозрачный объект, в котором располагается два других объекта - жидкость, стержень.
  
Жидкость --- это объект, который тоже является прозрачным тонкостенным цилиндром.

Стержень --- это непрозрачный прямоугольный параллелепипед. Служит для того чтобы отобразить на экране преломление твердого тела в жидкости.

Плоскость --- это некая ограничивающая плоскость. Предполагается, что под такой плоскостью не расположено никаких объектов. Располагается на минимальной координате по оси У. 

\section{Обоснование выбора формы задания трехмерных моделей}

Отображением формы и размеров объектов являются модели. 
Обычно используются три формы задания моделей.

\begin{enumerate}
	\item Каркасная (проволочная) модель.
	
	Одна из простейших форм задания модели, так как мы храним информацию только о вершинах и ребрах нашего объекта. Недостаток данной модели состоит в том, что она не всегда точно передает представление о форме объекта.
	
	\item Поверхностная модель.
	
	Поверхностная модель объекта --- это оболочка объекта, пустая внутри. Такая информационная модель содержит данные только о внешних геометрических параметрах объекта. Такой тип модели часто используется в компьютерной графике. При этом могут использоваться различные типы поверхностей, ограничивающих объект, такие как полигональные модели, поверхности второго порядка и др.
	
	\item  Объемная (твердотельная) модель.
	
	При твердотельном моделировании учитывается еще материал, из которого изготовлен объект. То есть у нас имеется информация о том, с какой стороны поверхности расположен материал. Это делается с помощью указания направления внутренней нормали.
	
\end{enumerate}

При решении данной задачи подойдут будет использоваться объемная модель. Этот выбор обусловлен тем, что каркасные модели могут привести к неправильному восприятию формы объекта, а поверхностные модели не подходят, так как важен материал из которого сделаны объекты сцены.

\section{Задание объемных моделей}

После выбора модели, необходимо выбрать лучший способ представления объемной модели.

Аналитический способ --- этот способ задания модели характеризуется описанием модели объекта, которое доступно в неявной форме, то есть для получения визуальных характеристик необходимо дополнительно вычислять некоторую функцию, которая зависит от параметра.

Полигональной сеткой --- данный способ характеризуется совокупностью вершин, граней и ребер, которые определяют форму многогранного объекта в трехмерной компьютерной графике.

Стоит отметить, что одним из решающих факторов в выборе способа задания модели в данном проекте является скорость выполнения преобразований над объектами сцены.

При реализации программного продукта представлением является аналитический способ, так как все объекты в сцене являются простыми геометрическими фигурами.

