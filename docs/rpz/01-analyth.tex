\chapter{Аналитический раздел}

В этом разделе будут представлены описание объектов, а также обоснован выбор алгоритмов, которые будут использован для ее визуализации.

\section{Описание объектов сцены}

Сцена состоит из источника света, цилиндра, жидкости, стержня и плоскости.

Источник света представляет собой материальную точку, пускающую лучи света во все стороны (если источник расположен в бесконечности, то лучи идут параллельно). Источником света в программе является вектор.
 
Цилиндр --- это тонкостенный прозрачный объект, в котором располагается два других объекта - жидкость, стержень.
  
Жидкость --- это объект, который тоже является прозрачным тонкостенным цилиндром.

Стержень --- это непрозрачный прямоугольный параллелепипед. Служит для того чтобы отобразить на экране преломление твердого тела в жидкости.

Плоскость --- это некая ограничивающая плоскость. Предполагается, что под такой плоскостью не расположено никаких объектов. Располагается на минимальной координате по оси У. 
