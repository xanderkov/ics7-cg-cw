\chapter*{Введение}
\addcontentsline{toc}{chapter}{Введение}

На сегодняшний день компьютерная графика является одним из самых быстрорастущих сегментов в области информационных технологий.

Область ее применения очень широка и не ограничивается художественными эффектами: в отраслях техники, науки, медицины и архитектуры трехмерные графические объекты используются для наглядного отображения разнообразной информации и презентации различных проектов.

Алгоритмы создания реалистичных изображений требуют особого внимания, поскольку они связаны с внушительным объемом вычислений, требуют больших компьютерных ресурсов и затратны по времени. Для создания качественного изображения объекта следует учесть не только оптические законы, но и расположение источника света, фактуры поверхностей. 

Основным направлением в развитии компьютерной графики является ускорение вычислений и создание более качественных изображений.

Цель курсового проекта --- разработка, реализация, описание программного обеспечения, генерирующее реалистичное изображение на основе стержня помещенного в цилиндр с жидкостью.   

\newpage

Задачи курсового проекта:

\begin{itemize}
	\item описать структуру трехмерной сцены, включая объекты, из которых она состоит;
	\item проанализировать существующие алгоритмы построения изображения и обосновать выбор тех из них, которые в наибольшей степени подходят для решения поставленной задачи;
	\item проанализировать и выбрать варианты оптимизации ранее выбранного алгоритма удаления невидимых линий;
	\item реализовать выбранные алгоритмы;
	\item разработать программное обеспечение для отображения сцены и визуализации стержня в цилиндре, наполненного жидкостью; 
\end{itemize}
